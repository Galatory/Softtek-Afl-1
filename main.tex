\documentclass{article}

\usepackage[utf8]{inputenc}
\usepackage[danish]{babel}
\usepackage{biblatex}
\usepackage{minted} %code highlighting

\title{Boblesortering}
\author{Nicolai Verbaarschot, s155932 \and Jakob Kristensen, s184228}
\date{1. oktober 2018}

\begin{document}

\maketitle
\begin{abstract}
    Dette dokument omhandler boblesortering. Der beskrives algoritmen og præsenteres en kompleksitetsanalyse.
\end{abstract}

\section{Introduktion}
Boblesortering (\textit{eng. bubble sort}) er en populær sorteringsalgoritme og er en af de simpleste algoritmer at forstå og implementere. Dog er den ikke en særlig effektiv sorteringsalgoritme1; hverken for store eller sm ̊a lister, og den anvendes sjældent i praksis. Boblesortering sorterer, som navnet antyder, elementerne i en liste ved at \textit{boble} hvert element gennem listen til sin rette plads i listen.

test placeholder

\subsection{Pseudokode}
Wikipedia~\cite{wiki} giver følgende pseudokode for boblesortering.

\begin{minted}{python}
procedure bubbleSort( A : list of sortable items ) defined as:
    do
        swapped := false
        for each i in 0 to length(A) - 2 inclusive do:
          if A[i] > A[i+1] then
            swap( A[i], A[i+1] )
            swapped := true
        end if end for
    while swapped
end procedure
\end{minted}

Antallet af sammenligninger, som boblesortering udfører paa en tabel af længde n, er i værste fald
En illustration af en kørsel af boblesortering fra Wikipedia kan ses på figur 1.

\section{Analyse af boblesortering}


\begin{equation}
\sum_{i=1}^{n-1} i=1+2+3+...+n-1=\frac{n(n-1)}{2}
\label{eq:sn}
\end{equation}

I bedste fald er antallet n-1

\begin{figure}
\centering
\includegraphics[width=1\textwidth]{raw_data.png}
\caption{\label{fig:data}Figur 1.}
\end{figure}

For en komplet introduktion til boblesortering og relaterede sorteringsalgoritmer se Knuth [1].




\section{Videre læsning}

\begin{thebibliography}{9}

    \bibitem{knuth} 
    Donald Knuth. The Art of Computer Programming, Volume 3. Addison Wesley. 
    
    \bibitem{wiki} 
    \texttt{http://en.wikipedia.org/wiki/Bubble\_sort}

\end{thebibliography}

\end{document}
